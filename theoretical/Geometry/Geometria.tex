\subsubsection{Geometria Básica}
\paragraph{Produto Escalar.}
Geometricamente é o produto do comprimento do vetor $a$ pelo comprimento da projeção do vetor $b$ sobre $a$.

$$a \cdot b = \|a\|\cos\theta \|b\| = x_1x_2+y_1y_2+z_1z_2$$

\paragraph{Propriedades.} \empty

	\begin{enumerate}
		\item $a \cdot b = b \cdot a$.
		\item $(\alpha \cdot a) \cdot b = \alpha\cdot(a \cdot b)$.
		\item $(a+b)\cdot c = a\cdot c + b \cdot c$.
		\item Norma de $a$ (comprimento ao quadrado): $\|a\|^2 = a \cdot a$.
		\item Projeção de $a$ sobre o vetor $b$: $\frac{a \cdot b}{\|b\|}$.
		\item Ângulo entre os vetores: $\cos^{-1}{\frac{a \cdot b}{\|a\|\|b\|}}$.
		\item $a \cdot b$ é negativo se o ângulo entre $a$ e $b$ é agudo, positivo se obtusoe igual à 0 eles formam um ângulo reto.	
	\end{enumerate}

\paragraph{Produto Vetorial.}
Dados dois vetores independentes linearmente $a$ e $b$, o produto vetorial $a \times b$ é um vetor perpendicular ao vetor $a$ e ao vetor $b$ e é a normal do plano contendo os dois vetores.

\begin{center}
	$
	a \times b = det
	\begin{vmatrix}
		e_x & e_y & e_z \\
		x_1 & y_1 & z_1 \\
		x_2 & y_2 & z_2 \\
	\end{vmatrix}, 
	$
	$
	a \cdot (b \times c) = det
	\begin{vmatrix}
		x_1 & y_1 & z_1 \\
		x_2 & y_2 & z_2 \\
		x_3 & y_3 & z_3 \\	
	\end{vmatrix}
	$
\end{center}

\paragraph{Propriedades.} \empty

	\begin{enumerate}
		\item $a \times b = -b \times a$.
		\item $(\alpha \cdot a) \times b = \alpha \cdot (a \times b)$.
		\item $a \cdot (b \times c) = b\cdot (c \times a) = -a\cdot(c\times b)$.
		\item $(a+b)\times c = a\times c + b\times c$.
		\item $\|a \times b \| = \|a\|\sin\theta \|b\|$.
		\item Volume do paralelepípedo formado por $a$, $b$ e $c$: $|a \cdot (b \times c)|$.
		\item Área do paralelogramo formado por $a$ e $b$: $|e_z \cdot (a \times b)| = |x_1y_2-y_1x_2|$.
		\item O sinal do coeficiente $e_z$ do produto vetorial indica a orientação relativa dos vetores. Se positivo, o ângulo de $a$ e $b$ é anti-horário. Se negativo, o ângulo é horário e se for zero, os vetores são colineares.
	\end{enumerate}

	\paragraph{Distância entre dois pontos.} Dados dois pontos $a = (x_1,y_2)$ e $b = (x_2,y_2)$, a distância entre $a$ e $b$ é dada por:
		$$d_{a,b} = \sqrt{(x_1 - x_2)^2 + (y_1 - y2)^2}$$

\paragraph{Condição de alinhamento de três pontos.} Dados três pontos $a = (x_1,y_2)$, $b = (x_2,y_2)$ e $c = (x_3,y_3)$, os pontos $a$, $b$ e $c$ estão alinhados se:
		$$det
			\begin{vmatrix}
				x_1 & y_1 & 1 \\
				x_2 & y_2 & 1 \\
				x_3 & y_3 & 1
			\end{vmatrix} = 0 $$

\paragraph{Equação da Reta (forma geral).} Os pontos $(x,y)$ que pertencem a uma reta $r$ devem satisfazer:
		$$ax + by + c = 0$$

\paragraph{Equação da Reta (forma reduzida).} A equação reduzida da reta, em que $m = \tan(a) = \frac{\Delta y}{\Delta x}$ é o coef. angular, e $n$ é o coef. linear, isto é, o valor de $y$ em que a reta intercepta o eixo $y$, é dada por:
		$$y = mx + n = m(x - x_0) +y_0$$

\paragraph{Distância entre ponto e reta.} Dados um pontos $p = (x_1,y_1)$ e uma reta $r$ de equação $ax+by+c=0$, a distância entre $p$ e $r$ é dada por:
		$$d_{p,r} = \frac{|ax_1+by_1+c|}{\sqrt{a^2+b^2}}$$

\paragraph{Interseção de retas.} Para determinar os pontos de interseção é necessário resolver o seguinte sistema de equações lineares:
$$
\begin{cases}
	a_1x + b_1y + c_1 = 0 \\
	a_2x + b_2y + c_2 = 0
\end{cases}
$$

A solução é dada por:
$$x = -\frac{c_1b_2 - c_2b_2}{a_1b_2 - a_2b_1}, \quad y = -\frac{a_1c_2 - a_2c_1}{a_1b_2 - a_2b_1}.$$

No caso do denominador for igual à 0, calculamos as seguintes determinantes, e se as duas são iguais à 0, as linhas são sobrepostas. Caso contrário, as linhas são paralelas e distintas.

\begin{center}
	$det
	\begin{vmatrix}
		a_1 & c_1 \\
		a_2 & c_2 \\
	\end{vmatrix}, 
	$
	$det
	\begin{vmatrix}
		b_1 & c_1 \\
		b_2 & c_2 \\
	\end{vmatrix}
	$
\end{center}

\paragraph{Equação da Circuferência (forma reduzida).} Os pontos $(x,y)$ que pertencem a uma circuferência $c$ devem satisfazer:
		$$(x-a)^2+(y-b)^2 = r^2,$$
		onde $(a,b)$ é o centro da circuferência e $r$ o seu raio.

\paragraph{Equação da Circuferência (forma geral).} A partir da equação reduzida da circuferência, encontramos a equação geral:
		$$x^2 +y^2-2ax-2by+(a^2+b^2-r^2)=0$$

\paragraph{Interseção entre reta e circuferência.} Para determinar o tipo de interseção é necessário resolver o seguinte sistema não-linear: 
$$
\begin{cases}
	ax+by+c = 0 \\
	x^2 +y^2-2ax-2by+(a^2+b^2-r^2)=0
\end{cases}
$$

Há três possibilidades como solução do sistema:
		\begin{enumerate}
			\item Reta exterior à circuferência: nenhuma solução. A reta não possui nenhum ponto de comum com a circuferência.
			\item Reta tangente à circuferência: solução única. A reta possui apenas 1 ponto em comum com a circuferência.
			\item Reta secante à circuferência: duas soluções. A reta cruza a circuferência em 2 pontos distintos.
		\end{enumerate}

\subsubsection{Geometria Plana}

\paragraph{Triângulos.} Polígono com três vértices e três arestas. Uma aresta arbitrária é escolhida como a base e, nesse caso, o vértice oposto é chamado de ápice. Um triângulo com vértices $A$, $B$ e $C$ é denotado $\triangle ABC$.

	\begin{itemize}
		\item Comprimento dos lados: $a,b,c$
		\item Semiperímetro: $p = \frac{a+b+c}{2}$
		\item Altura:
		\begin{itemize}
			\item Equilátero: $h = \frac{\sqrt{3}}{2}l$
			\item Isósceles: $h = \sqrt{l^2 - \frac{b^2}{4}}$
		\end{itemize}
		\item Área:
		\begin{itemize}
			\item Equilátero: $A =\frac{l^2\sqrt{3}}{4}$
			\item Isósceles: $A =\frac{1}{2} bh$
			\item Escaleno: $A =\sqrt{p(p-a)(p-b)(p-c)}$
		\end{itemize}
		\item Raio circunscrito: $R = \frac{1}{4A}abc$
		\item Raio inscrito: $r = \frac{1}{p}A$
		\item Tamanho da mediana: $m_a = \frac{1}{2}\sqrt{2b^2+2c^2-a^2}$
	\end{itemize}

\noindent \textbf{Quadriláteros.} Polígono de quatro lados, tendo quatro arestas e quatro vértices. Um quadrilátero com vértices $A$, $B$, $C$ e $D$ é denotado com $\square ABCD$.

	\begin{itemize}
		\item Comprimento dos lados: $a,b,c,d$
		\item Semiperímetro: $p = \frac{a+b+c+b}{2}$
		\item Área:
		\begin{itemize}
			\item Quadrado: $a^2$
			\item Retângulo: $b\cdot h$
			\item Losango: $\frac{1}{2}D \cdot d$
			\item Trapézio: $\frac{1}{2}h(B+b)$
		\end{itemize}
		\item Perímetro:
		\begin{itemize}
			\item Quadrado: $4a$
			\item Retângulo: $2(b+h)$
			\item Losango: $4a$
			\item Trapézio: $B+b+l_1+l_2$
		\end{itemize}
		\item Diagonal:
		\begin{itemize}
			\item Quadrado: $a\sqrt{2}$
			\item Retângulo: $\sqrt{b^2 + h^2}$
			\item Losango: $a\sqrt{2}$
			\item Trapézio: $\sqrt{h^2 + \frac{(B-b)^2}{4h}}$
		\end{itemize}
	\end{itemize}

\paragraph{Círculos.} Forma que consiste em todos os pontos de um plano que estão a uma determinada distância de um ponto dado, o centro. A distância entre qualquer ponto do círculo e o centro é chamada de raio.

	\begin{itemize}
		\item Área: $A = \pi r^2$
		\item Perímetro: $C = 2\pi r$
		\item Diâmetro: $d = 2r$
		\item Área do setor circular: $A = \frac{1}{2}r^2\theta$
		\item Comprimento do arco: $L = r\theta$
		\item Perímetro do setor circular: $P = r(\theta+2)$
	\end{itemize}

\paragraph{Teorema de Pick.} Suponha que um polígono tenha coordenadas inteiras para todos os seus vértices. Seja $i$ o número de pontos inteiros no interior do polígono e $b$ o número de pontos inteiros na sua fronteira (incluindo vértices e pontos ao longo dos lados). Então, a área $A$ deste polígono é:
$$A = i + \frac{b}{2} -1.$$
$$b = \gcd(|x_2-x_1|,|y_2-y_1|)+1.$$

\subsubsection{Geometria Espacial}


	\begin{itemize}
		\item Área da Superfície:
		\begin{itemize}
			\item Cubo: $6a^2$
			\item Prisma: $A_l + 2A_b$
			\item Esfera: $4\pi r^2$
			\item Cilindro: $2\pi r(h+r)$
			\item Cone: $\pi r(r+\sqrt{h^2+r^2})$
			\item Pirâmide: $A_b+\frac{1}{2}P_b \cdot g, \quad g=\textnormal{geratriz}$
		\end{itemize}

		\item Volume:
		\begin{itemize}
			\item Cubo: $a^3$
			\item Prisma: $A_b\cdot h$
			\item Esfera: $\frac{4}{3} \pi r^3$
			\item Cilindro: $\pi r^2h$
			\item Cone: $\frac{1}{3} \pi r^2 h$
			\item Pirâmide: $\frac{1}{3} A_b\cdot h$
		\end{itemize}

	\end{itemize}


\paragraph{Fórmula de Euler para Poliedros.} Os números de faces, vértices e arestas de um sólido não são independentes, mas estão relacionados de uma maneira simples.
$$F + V - A = 2.$$

\subsubsection{Trigonometria}
\paragraph{Funções Trigonométricas.}
	$$\sin\theta = \frac{\textnormal{cateto oposto a }\theta}{\textnormal{hipotenusa}}  \quad \cos\theta = \frac{\textnormal{cateto adjacente a }\theta}{\textnormal{hipotenusa}} \quad \tan\theta = \frac{\textnormal{cateto oposto a }\theta}{\textnormal{cateto adjacente a}\theta}$$

\paragraph{Ângulos notáveis.}

\begin{center}
	\begin{tabular}{|c|c c c c c|}
		$\theta$ & $0^{\circ}$ & $30^{\circ}$&$45^{\circ}$&$60^{\circ}$&$90^{\circ}$ \\
		\hline
		\rule{0pt}{0.5cm}$\sin\theta$ &0&$\frac{1}{2}$&$\frac{\sqrt{2}}{2}$&$\frac{\sqrt{3}}{2}$&1 \\
		\hline
		\rule{0pt}{0.5cm}$\cos\theta$ &1&$\frac{\sqrt{3}}{2}$&$\frac{\sqrt{2}}{2}$&$\frac{1}{2}$&0 \\
		\hline
		\rule{0pt}{0.5cm}$\tan\theta$ &0&$\frac{\sqrt{3}}{3}$&1&$\sqrt{3}$&$\infty$ \\
		
	\end{tabular}
\end{center}

\textbf{Propriedades.}

	\begin{enumerate}
		\item $\sin(a+b) = \sin a \cos b + \cos a \sin b$
		\item $\cos(a+b) = \cos a \cos b - \sin a \sin b$
		\item $\tan(a+b) = \frac{\tan a + \tan b}{1 - \tan a \tan b}$
		\item $a\sin x+b\cos x = r\sin(x+\phi)$, onde $r=\sqrt{a^2+b^2}$ e $\phi = \tan^{-1}\frac{b}{a}$ 
		\item $a\cos x+b\sin x = r\cos(x-\phi)$, onde $r=\sqrt{a^2+b^2}$ e $\phi = \tan^{-1}\frac{b}{a}$
		\item \textbf{Lei dos Senos}: $$\frac{a}{\sin \hat{A}} = \frac{b}{\sin \hat{B}} = \frac{c}{\sin \hat{C}} = 2r.$$
		\item \textbf{Lei dos Cossenos}:
		$$a^2 = b^2 +c^2+2\cdot b\cdot c\cdot\cos\hat{A}$$
		$$b^2 = a^2 +c^2+2\cdot a\cdot c\cdot\cos\hat{B}$$
		$$c^2 = b^2 +a^2+2\cdot b\cdot a\cdot\cos\hat{C}$$
		\item \textbf{Teorema de Tales}: A interseção de um feixe de retas paralelas por duas retas transversais forma segmentos proporcionais:
		$$\frac{\overline{AB}}{\overline{BC}} = \frac{\overline{DE}}{\overline{EF}}$$
		\item \textbf{Casos de semelhança}: dois triângulos são semelhantes se
		\begin{itemize}
			\item dois ângulos de um são congruentes a dois do outro. Critério AA (Ângulo, Ângulo).
			\item os três lados são proporcionais aos três lados do outro. Critério LLL (Lado, Lado, Lado).
			\item possuem um ângulo congruente compreendido entre lados proporcionais. Critério LAL (Lado, Ângulo, Lado).
		\end{itemize}

	\end{enumerate}
