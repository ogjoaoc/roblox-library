\subsubsection{Propriedades do Grau dos Vértices}
\begin{itemize}
    \item A soma de todos os graus dos vértices do grafo é igual a 
    \[
    \sum_{i=1}^{n} d(v_i) = 2m
    \]
    \item A soma dos graus é sempre um número par.
    \item O número de vértices de grau ímpar em um grafo é sempre par.
\end{itemize}

\subsubsection{Passeio, Trilha, Caminho e Ciclo}
\begin{itemize}
    \item \textbf{Passeio}: sequência de vértices onde cada par consecutivo está conectado.
    \item \textbf{Trilha}: passeio sem repetição de arestas.
    \item \textbf{Caminho}: trilha sem repetição de vértices.
    \item \textbf{Ciclo}: caminho fechado com comprimento $\ge2$.
\end{itemize}

\subsubsection{Caminhos}
\subsubsection*{Caminho de Euler} 
Um caminho de Euler em um grafo é o caminho que visita cada aresta exatamente uma vez. Um ciclo de Euler, ou Tour de Euler, em um grafo é um ciclo que usa cada aresta exatamente uma vez.

\paragraph{Teorema:} Um grafo conectado tem um ciclo de Euler se, e somente se, cada vértice possui grau par.

\subsubsection*{Caminho Hamiltoniano} 
Um caminho Hamiltoniano em um grafo é o caminho que visita cada vértice exatamente uma vez. Um ciclo Hamiltoniano em um grafo é um ciclo que visita cada vértice exatamente uma vez.

\paragraph{Teoremas:} \empty
\begin{itemize}
    \item Teorema de Dirac: Um grafo simples com $n$ vértices $(n\ge 3)$ é Hamiltoniano se cada vértice tem grau $ \ge \frac{n}{2}$.
    \item Teorema de Ore: Um grafo simples com $n$ vértices $(n\ge 3)$ é Hamiltoniano se, para cada par de vértices não-adjacentes, a soma de seus graus é $\ge n$.
    \item Ghouila-Houiri: Um grafo direcionado simples fortemente conexo com $n$ vértices é Hamiltoniano se cada vértice tem um grau $\ge n$.
    \item Meyniel: Um grafo direcionado simples fortemente conexo com $n$ vértices é Hamiltoniano se a soma dos graus de cada par de vértices não-adjacentes é $\ge 2n-1$.
\end{itemize}

\subsubsection{Árvores}
\subsubsection*{Propriedades Fundamentais}
\begin{itemize}
    \item Todo par de vértices tem exatamente \textbf{um caminho simples} único entre eles.
    \item Remover qualquer aresta desconecta o grafo.
    \item Adicionar qualquer aresta cria exatamente um ciclo.
    \item Fórmula de Euler: Para $n$ vértices, possui exatamente $n-1$ arestas.
\end{itemize}

\subsubsection*{Teoremas Importantes}
\begin{enumerate}
    \item \textbf{Centro de Árvore}: Toda árvore tem no máximo 2 centros (vértices de excentricidade mínima).
    \item \textbf{Diâmetro}: Maior distância entre dois vértices (calculável via 2 BFS).
    \item \textbf{Encaminhamento Único}: Para qualquer vértice $r$ (raiz), existe orientação natural das arestas.
\end{enumerate}

\subsubsection{Árvore Geradora Mínima (AGM)}
\subsubsection*{Propriedades de AGM}
\begin{itemize}
\item \textbf{Unicidade}: Se pesos são distintos, a AGM é única.
\item \textbf{Ciclo}: Para qualquer ciclo $C$ em $G$, a aresta mais pesada de $C$ não está na AGM.
\item \textbf{Corte}: Para qualquer corte $(S, V\setminus S)$, a aresta mais leve cruzando o corte pertence à AGM.
\end{itemize}